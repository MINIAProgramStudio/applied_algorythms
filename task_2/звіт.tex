\documentclass{article}
\usepackage[utf8]{inputenc}
\usepackage[ukrainian]{babel}
\PassOptionsToPackage{hyphens}{url}\usepackage{hyperref}
\title{Прикладні алгоритми. Завдання 2, звіт}
\author{Михайло Голуб}
\usepackage{graphicx}
\graphicspath{ {./img/} }
\begin{document}
\maketitle
\newpage

\textbf{Реалізація класу граф:}
\\\indent
Створено клас \textit{Graph} що при ініціалізації приймає на вхід матрицю ребер. Цей клас має наступні методи:
\begin{itemize}
\item[] \textbf{\textit{to\_ll}} -- повертає масив списків зв'язності;
\item[] \textbf{\textit{clear}} -- встановлює усі клітинки матриці в False;
\item[] \textbf{\textit{from\_ll}} -- створює нову матрицю зв'язності з масиву списків зв'язності;
\item[] \textbf{\textit{add\_vertice}} -- додає одну колонку та один рядок в матрицю суміжності;
\item[] \textbf{\textit{add\_edge}} -- встановлює відповідну пару клітинок в True;
\item[] \textbf{\textit{remove\_vertice}} -- видаляє колонку та рядок з вказаним індексом;
\item[] \textbf{\textit{remove\_edge}} -- встановлює відповідну пару клітинок в False.\\\\
\end{itemize}

\textbf{Реалізація інших класів:}\\\indent
Клас \textit{OrientedGraph} є нащадком (не дитиною, від англ. inherit -- наслідувати / успадковувати) \textit{Graph} з перевизначенням додання та видалення ребер: значення встановлюється не для пари клітинок, а для однієї клітинки.\\\indent
Клас \textit{WeightedGraph} є нащадком \textit{Graph} з перевизначенням методів: методи працюють не з булевими значеннями, а з числовими.\\\indent
Клас \textit{OrientedWeightedGraph} є нащадком \textit{WeightedGraph} з перевизначенням додання та видалення ребер: значення встановлюється не для пари клітинок, а для однієї клітинки.\\\\\indent
\textbf{Реалізація рандомізованих класів:}\\\indent
Класи \textit{RandomGraph, RandomOrientedGraph, RandomWeightedGraph, RandomOrienedWeightedGraph} є нащадками відповідних класів з перевизначенням \textit{\_\_init\_\_}: створюється пуста (заповнена False) матриця суміжності потрібного розміру, і через неї проходить цикл що з вказаною ймовірністю записує True або випадкове число. Для того щоб в не орієнтованих графах кожне ребро було пройдено циклом лише один раз, ітератор другої координати працює від i до n, де i -- позиція ітератора першої координати та n -- кількість вершин. В орієнтованих графах обидва ітератори працюють від 0 до n
\end{document}